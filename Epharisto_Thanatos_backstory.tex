\documentclass[10pt,twoside,twocolumn,openany]{book}
\usepackage[bg-letter]{dnd} % Options: bg-a4, bg-letter, bg-full, bg-print, bg-none.
\usepackage[english]{babel}
\usepackage[utf8]{inputenc}

% Start document
\begin{document}
\fontfamily{ppl}\selectfont % Set text font

% Your content goes here

\chapter{Epharisto Thanatos}
\begin{figure}
  \includegraphics[width = 0.4\textwidth]{ET.png}
\end{figure}

\begin{monsterbox}{Epharisto Thanatos}
	\textit{Human, Lvl. 5 Cleric of Nerull (Grave Domain), true neutral}\\
	\hline
	\basics[%
	armorclass = 18,
	hitpoints  = 38 (5d8 + 10),
	speed      = 30 ft
	]
	\hline
	\stats[
    STR = \stat{9}, % This stat command will autocomplete the modifier for you
    DEX = \stat{14},
    CON = \stat{14},
    INT = \stat{12},
    WIS = \stat{18},
    CHA = \stat{12}
  ]
	\hline
	\details[%
	% If you want to use commas in these sections, enclose the
	% description in braces.
	% I'm so sorry.
  savingthrows = {WIS $+7$, CHA $+4$},
  skills = {Passive Perception $17$, Passive Investigation $11$},
	languages = {Common, 3 others},
	]
	\hline \\[1mm]
	\begin{monsteraction}[Shelter the Faithful]
		Epharisto and his party can receive free healing at temples and shrines that recognize him as a cleric of Nerull. In addition, Epharisto commands respect from worshippers of Nerull and he is able to perform the rituals and rites associated with the Church of Nerull.
	\end{monsteraction}\\
  \begin{monsteraction}[Lucky]
    3/day, Epharisto feels the universe turning in his favor.
    Any attack, saving throw, or skill check that he makes can be rerolled after the die is rolled but before the result is determined.
    Epharisto can choose which roll to use.
    This also applies to attack rolls to hit Epharisto.
  \end{monsteraction}

\end{monsterbox}

\section{Early life}
\begin{itemize}
\item No memory of life before the temple of Nerull
\item Raised to be a cleric of the God
\item Primary teacher was Charis, an older blind cleric who claimed to have blinded himself after seeing the face of Nerull so that his God's face would be the last he ever remembered.
\item Temple was attacked when he was 15
\end{itemize}

Epharisto grew up curious. Fortunately, Charis was kind as clerics of Nerull go and indulged the curiosity.
This led to the early fascination of death and the veil between worlds.
His fascination kept him an outcast from the other clerics, but Charis did a good job of protecting and deflecting the attention from the other ``more orthodox'' clerics in the temple. \\

A force of undead led by an unknown necromancer attacked the temple of Nerull in Epharisto's 15$^\textrm{th}$ year.
This attack almost razed the temple and Charis was killed in the fighting.
Epharisto recieved massive burns on his entire body defending and retriving Charis's body.
He succummbed to the burns but was revivified; however, all the hair on his body was burned off and it has never grown back, despite a number of resotration spells that have removed all traces of scarring.
He dreams of fire and and of the abomination that is the undead from time to time.
In these dreams, he hears a voice commanding him to destroy the undead and comanding him to send the chosen to his God's embrace.
He thinks this voice is Nerull but he never saw a face and no other cleric beyond Charis had ever had contact with Nerull.
\begin{commentbox}{Epharisto's quest --- The veil between life and death}
  Epharisto is driven by a desire to understand death.
  He wants to understand the nature of death --- how is it that someone can travel past the veil and come back with \emph{Raise Dead}?
  What happens to their memories?
  How is being raised different than necromancy (\emph{Ugh\ldots gross!!})?
  What about \emph{Revivify}?
  How does Nerull know when it is time for someone to die; i.e.\ why is it ok to heal some people but send others straight to His arms?
  What is Nerull like?
  Charis never talked about pain being something Nerull cared about; however, pain was all Mavrokeras ever talked about.
  How does Nerull really feel?
\end{commentbox}

\section{The middle years}
\begin{itemize}
  \item Fanatic elemenets in the temple came to power in response to the attack
  \item Head Cleric Mavrokeras advocates for hastening the living towards the realm of death through pain and suffering.
  \item Mavrokeras puts a moratorium on all healing and all ressurection spells.
  \item Epharisto is ``asked'' to take a sabbatical into the world due to his open agitation and dissent in the temple.
  \item Upon leaving the temple, Epharisto found a silver bell with no clapper hanging from a tree that was mysteriously ringing\ldots
\end{itemize}
\begin{commentbox}{Epharisto's trinket --- the silver bell}
  This silver bell has no clapper and was found handing from a tree.
  Regardless, it is always shiny and strangely can be heard ringing from time-to-time. From the player's perspective, the bell will ring whever Epharisto heals someone from 0 HP.\@
  It will also ring when Epharisto uses his Channel Divinity feature.
  From the DM's perspective/a story perspective, the bell rings whenever Nerull feels that someone Epharisto is near to should die.
  This choice is alwawys obvios and so far, Epharisto does not feel like he has sent the wrong person to Nerull.
  The bell can also rig as thematic points when Nerull wished to make his presence known or his voice heard.
\end{commentbox}
After the attack, fanatic elements gained power in the temple behind their leader, the charismatic Mavrokeras.
Mavrokeras and his followers believe that any action that takes a person further from death is sacriledge and just as bad necromancy.
To the fanatics, life is an inferior state and the temple's purpose is to hasten the living along their journey to Nerull's kingdom.
Epharisto's vocal dissent and agitation led to his ostratization by the fanatics.
Eventually, fed up with what he believes is nonsense, Epharisto openly accused Mavrokeras of turning the temple into a ``cult of pain with no respect for the life that preceeds death'' that now exists to ``pervert Nerull's will with masichistic machinations''. \\

Soon after his outbust, Mavrokeras ``advised'' Epharisto that he might better serve Nerull outside the temple as a traveling cleric.
As the writing on the wall was clear, Epharisto left the temple the next morning; however, as he left, Mavrokeras staged a scene and demanded the skull from Epharisto's holy symbol as he was no longer ``suitable representive of death's will''.
A mile or two down the road, farther away from the temple than he had ever been, Epharisto saw a silver bell hanging from a tree, ringing in the wind.
However, as he handled the bell to inspect it, the ringing stopped and he found that it had no clapper.
He took this as a sign and hung the bell from his scythe, taking the place of the skull Mavrokeras demanded.
Epharisto has found that the bell rings whenever he heals someone from unconsious, and whenever he channels his divinity.
Sometimes he can even hear the bell ringing moments before he gets the feeling that Nerull demands that someone die.

\begin{paperbox}{Epharisto's appearance}
  Epharisto is tall and gaunt with piercing ``too-wide'' light blue eyes and a complete absence of hair.
  He wears a dark black garmants with a blood-red sash about the waist.
  His armor is dull and worn over the garments.
  He often wears his vestments, a dark black robe with a hood and deep sleeves over his garments/armor.
  The robe never seems to fade, no matter how often Epharisto fastidiously washed them.
  He is unsure whether this is because of Nerull's unfluence of if the temple posesses some deep knowledge of color and dye.
  He  carries his holy symbol, a scythe with a silver bell with no clapper hung from the top of the shaft.
  The scythe is ceremonial in nature and is dull to the touch.
  He carries a dagger in an easily concealed forearm sheath for times when the action becomes a bit too personal.


\end{paperbox}
\section{Post-exile}
\begin{itemize}
  \item For 2 years, Epharisto has been traveling and working to determine the nature of veil between worlds.
  \item Along the way, he has linked up with different characters, typically traveling with those whom bodies tend to follow (or even precede)
  \item Epharisto is happy to provide healing, but he likes to withold it as long as possible and will ofen interrogate his patients aftwerwords about their experience
\end{itemize}
Since the exile 2 years prior, Epharisto has traveled and become somewhat of an adventurer.
This is purely accidental --- his goal is to investigate death, and death is a common companion of many adventurers.
He has taken quite a liking to healing the injured, but only from the point of near death.
As ``payment'' for the healing, he likes to quiz his subject about their near-death experience.
Sometimes, his patients words are punctuated with the ringing of his bell.
His most recent companion is a investigator extrodinaire by the name of Hercule Gently.
The investigator has quite the reputation for being able to solve any mystery; however, while he may be able to say \emph{whodunit}, he has yet to provide a satisfying answer to any of Epharisto's questions.\\


\end{document}
